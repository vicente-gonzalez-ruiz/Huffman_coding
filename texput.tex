\title{Huffman coding}
\author{Vicente González Ruiz}
\maketitle
\tableofcontents

\section{\href{https://en.wikipedia.org/wiki/Huffman_coding}{Basics}}
\begin{itemize}
  \tightlist
\item Developed by D. Huffman in 1952~\cite{huffman1952method}.
\item
  (Absolute) Optimal performance (in average, better than Shannon-Fano)
  when a integer number of bits is assigned to each symbol.
\item
  Huffman-based VLC codecs build a binary tree where the symbols are
  stored in the leafs and the distance of each symbol to the root of the
  tree is \[\lceil\log_2(p(s))\rceil\].
\item
  After label each binary branch in the tree, the Huffman code-word for
  the symbol \(s\) is the sequence of bits (labels) that we must use to
  travel from the root to the \(s\)-leaf.
\end{itemize}

\section{Building Huffman trees}
\begin{enumerate}
\tightlist
\item
  Create a list of nodes. Each node stores a symbol and its probability.
\item
  While the number of nodes in the list \textgreater{} 1:
  \begin{enumerate}
  \tightlist
  \item
    Extract from the list the 2 nodes with the lowest probability.
  \item
    Insert in the list a new node (that is the root of a binary tree)
    whose probability is the sum of the probability of its leafs.
  \end{enumerate}
\end{enumerate}

\subsection{Example}
\svgfig{graphics/huffman_ejemplo}{6cm}{600}

\section{Encoder~\cite{nelson96datacompression}}
\begin{enumerate}
\tightlist
\item
  n := \textbar{}C\textbar{};
\item
  Q := C;
\item
  \textbf{for} i := 1 \textbf{to} n-1 \textbf{do}
  \begin{enumerate}
  \setcounter{enumii}{3}
  \tightlist
  \item
    allocate a new node z
  \item
    z.left := x := Extract-min(Q);
  \item
    z.right := y := Extract-min(Q);
  \item
    z.freq := x.freq + y.freq;
  \item
    Insert(Q,z);
  \end{enumerate}
\item
  \textbf{end for}
\item
  \textbf{return} Extract-min(Q); \{return the root of the tree\}
\end{enumerate}

\subsection{Example}

String: BACADAEAFABBAAAGAH

\begin{tabular}{ccccccccc}
  A & B & C & D & E & F & G & H\\
  \hline
 Frequency (in thousands) & 50 & 20 & 5 & 5 & 5 & 5 & 5 & 5 \\
 Fixed-length codeword & 000 & 001 & 010 & 011 & 100 & 101 & 110 & 111 \\
 Variable-length codeword & 0 & 100 & 1010 & 1011 & 1100 & 1101 & 1110 & 1111
\end{tabular}

\pngfig{graphics/huffman_encoder}{8cm}{800}

OUTPUT: 10001010010110110001101010010000111001111

\section{Decoder}
\begin{enumerate}
\tightlist
\item
  prob\_table := freq //Array with the frequency of occurrence of each
  char
\item
  sec := str // Output that we want to rebuild
\item
  tree := array //A key-value associating array, with each item's key
  the node in that subtree and value its current sum probability,
  according to prob\_table
\item
  code\_table := table //An array with length of the how many characters
  in prob\_table, wich maps each code to its corresponding decoded
  character with each code initialised as empty strings.
\item
  \textbf{while} more than one node in tree \textbf{do} sub\_nodes :=
  key of node with smallest probability in \emph{tree sec\_sub\_nodes} =
  key of := key of node with second smallest propability in \emph{tree}

  \begin{enumerate}
  \def\labelenumii{\arabic{enumii}.}
  \setcounter{enumii}{5}
  \tightlist
  \item
    \textbf{for} node in sub\_nodes \textbf{do}
    \begin{enumerate}
    \def\labelenumiii{\arabic{enumiii}.}
    \setcounter{enumiii}{6}
    \tightlist
    \item
      code\_table{[}node{]}.key := ``1'' + code\_table{[}node{]}.key;
    \end{enumerate}
  \item
    \textbf{end for}
  \item
    \textbf{for} node in sec\_sub\_nodes \textbf{do}

    \begin{enumerate}
    \def\labelenumiii{\arabic{enumiii}.}
    \setcounter{enumiii}{9}
    \tightlist
    \item
      code\_table{[}node{]}.key = ``0'' + code\_table{[}node{]}.key
    \end{enumerate}
  \item
    \textbf{end for}
  \item
    tree{[}sub\_nodes + sec\_sub\_nodes{]} := tree{[}sub\_nodes{]} +
    tree{[}sec\_sub\_nodes{]}
  \item
    delete tree{[}sub\_nodes{]}
  \item
    delete tree{[}sec\_sub\_nodes{]}
  \end{enumerate}
\item
  \textbf{end while}
\item
  temp := ""
\item
  result := ""
\item
  \textbf{for} code in seq \textbf{do}

  \begin{enumerate}
  \def\labelenumii{\arabic{enumii}.}
  \setcounter{enumii}{18}
  \tightlist
  \item
    temp := temp + code
  \item
    \textbf{if} temp in code\_table \textbf{then}

    \begin{enumerate}
    \def\labelenumiii{\arabic{enumiii}.}
    \setcounter{enumiii}{20}
    \tightlist
    \item
      result := result + code\_table{[}temp{]}
    \item
      temp := ""
    \end{enumerate}
  \item
    \textbf{end if}
  \end{enumerate}
\item
  \textbf{end for}
\item
  return result
\end{enumerate}

\subsection{Example}
\jpgfig{graphics/huffman_decoder}{12cm}{1200}
INPUT: \textbf{BACADAEAFABBAAAGAH}

\section{Limits}
\begin{itemize}
\item
  Any Huffman code satisfies that \begin{equation}
    l\big(c(s)\big) = \lceil I(s)\rceil, \tag{Eq:Huffman}
  \end{equation} where \(l\big(c(s)\big)\) is the length of the
  code-word assigned to the symbol \(s\). This implies that, with each
  encoded symbol, up to 1 bit of redundant data can be introduced (think
  about a very frequent -- high probability -- symbol).
\item
  This is a problem that grows when the size of the alphabet is small.
  In the extreme case, for binary source alphabets, the Huffman coding
  does not change the length of the original representation.
\end{itemize}

\bibliography{text-compression}
